% Preamble 
\documentclass{article} %This is saying its an article
\title{Accounting for the Correlation between Perceived Risks and Rewards to Crime}
\author{Memo: Rebecca Bucci}
%\date{} %Can also leave this out to date it with the current date
\usepackage{hyperref}
\usepackage{amsmath}
\usepackage[parfill]{parskip} %makes new paragraphs not indented
\usepackage[a4paper, total={7in, 8in}, margin=.55in]{geometry}
\usepackage{url} 
\usepackage[backend=biber, style=apa, sortcites=true, sorting=nyt]{biblatex}  
%\addbibresource{.bib}

\usepackage{booktabs}
\usepackage[dvips]{graphicx}
\usepackage{color}
\usepackage{geometry}
\usepackage{pdflscape}
\usepackage{mathtools}

\hypersetup{colorlinks=true, linkcolor=blue, urlcolor=blue, citecolor=blue} % This sets the font color of certain hyperlinks in a document


% Body
\begin{document}
\maketitle

%\tableofcontents


%KEYWORDS: 


%This research was funded via the Time-Sharing Experiments in the Social Sciences (TESS).

%Direct correspondence to Rebecca Bucci, Department of Sociology, Harvard University, 33 Kirkland  (email: rebeccabucci@fas.harvard.edu).

%ACKNOWLEDGMENTS:
%I would like to thank Dr. Jeremy Freese and Dr. Jamie Druckman
%I would also like to thank Tom Loughran for helpful feedback at the very earliest stages of this project. 
%Thank Justin Pickett after accepted somewhere?








\section{INTRODUCTION}

Perceived risks and rewards to crime are known to predict later offending and as such, are often the target of crime-reduction interventions. The negative relationship between risks and crime, as well as the positive relationship between rewards and crime, have long been established in criminology, however, individuals’ confounding of risk and rewards at the time of decision-making has largely been overlooked. The correlation between risk and reward (i.e., benefit) has been succinctly explained by Slovic and colleagues (2004: 315) who stated, “whereas risk and benefit tend to be positively corelated in the world, they are negatively correlated in people’s minds (and judgements).” In regard to crime, it is typically the case than higher rewards are related to higher risks (e.g., a bank robbery), but the aforementioned logic would lead us to believe that potential offenders may associate higher levels of reward with lower levels of risk. Similarly, potential offenders may associate lower levels of risk with higher levels of reward. If it is the case that individuals’ perceptions of these attributes of risk or reward are impacted by perceptions of the other, then this has important implications for policy solutions to crime. 

This hypothesis can best be explained by the affect heuristic. The origin of the affect heuristic begins from the finding that, despite the general pattern of a positive correlation between risk and benefit, individuals perceive negative correlations (Fischhoff et al., 1978). As scholars began to delve into this finding, several reached the conclusion that this association is due to the feeling or affect attached to the topic (Alhakami & Slovic, 1994; Finucane et al., 2000), a phenomenon that has since been referred to as the affect heuristic (Slovic et al., 2004). The implication of the affect heuristic is described by Slovic and colleagues (2004: 315); “If a general affective view guides perceptions of risk and benefit, providing information about benefit should change perception of risk and vice versa” (emphasis in original). If it is possible that parallels can be drawn to crime, where perceptions of rewards to crime impact perceptions of risk and vice versa, then this is an important implication for our understanding of offender decision-making.

Notably, recent criminological research has called for attention to behavioral economics, particularly cognitive heuristics and biases such as the affect heuristic, when examining crime and developing policy solutions to crime (Loughran, 2019; Pickett & Roche, 2016). Critiques of the offender decision-making literature often discuss how decisions are likley not as deliberate or thought-out as theories often suggest. However, heuristics offer one avenue for how potential offenders think about crime but do so quickly and automatically (a process described as a System 1 process in the psychological literature). Some have begun to theorize about the possibility that offenders downplay risks when rewards are high (Clark, 2013), however, few studies have tested this hypothesis or related hypotheses regarding cognitive biases in criminal decision-making (e.g., Pickett, 2018; Pogarsky & Loughran, 2016; Pogarsky, Roche & Pickett, 2018). 

If it is the case that rewards to crime directly impact one’s perceptions of risk, then the existing research on the role of risk and reward in offender decision-making may have masked the importance of risk, as well as the importance of the interplay between risk and reward. Though relatively fewer studies of offender decision-making have focused on the effect of rewards to crime relative to the risks (Loughran, 2019), the existing studies have generally concluded that rewards to crime are equally as important as perceived risks, or in some cases, are more important. As stated, these studies may be underestimating the effect of risk by ignoring the negative correlation between rewards and risks that occurs before respondents report on their perceived levels of risk and reward. If this is the case, recommendations to move crime-reduction strategies away from a focus on risks to a sole focus on reward may be less successful than expected given this nuance in decision-making. Additionally, crime-reduction strategies designed solely to impact perceptions of risk may be misguided if not coupled with efforts to decrease rewards.



\section{AFFECT HEURISTIC}

Evidence in favor of the affect heuristic been shown empirically by Finucane and colleagues (2000) who set out to examine whether individuals’ perceptions of the risk and rewards of technology were altered by manipulating either risk (high/low) or rewards (high/low).  The manipulations of risk were designed to alter affect by describing the severe risks of each technology (high risk) or by describing the lack of risks and safety protocols surrounding each technology (low risk). The manipulations of rewards alternatively described how beneficial the technology would be for society (high reward) or discussed the very limited benefits of the technology (low reward).

The results of this study suggest that affect is the mediator which explains the relationship between the impact of reward (risk) on perceptions of risk (reward) (Finucane et al., 2000; Slovic et al., 2004). In other words, being presented with information on the rewards of a new technology results in positive changes to one’s affect, subsequently resulting in decreased perceptions of risk. 

\section{AFFECT HEURISTIC IN CRIMINOLOGY}

The most applicable test of the relationship between reward and risk was conducted by Pogarsky and colleagues (2017) who examined how affect impacts perceived risks of being caught and the perceived benefits of texting while driving. In this study, affect was altered by presenting information on deaths from texting and driving (negative), information on the prevalence of texting (neutral), and the benefits from texting (positive). This study yielded results consistent with the affect heuristic for risk (i.e., risk was lowest in the positive group and highest in the negative group), but this pattern was not found for reward (Pogarksy, Roche & Pickett, 2017). A related study was conducted by Kamerdze and colleagues (2014) who examined the effect of recalling a negative, positive or neutral experience on intentions to drink and drive, or to cheat on an exam. Positive and negative affect were not found to be significantly related to decisions to offend; only when intense positive affect was examined by coding only those with the highest affect scores did positive affect negatively predict intentions to offend. Interestingly, perceptions of one’s risk of getting caught were not significant mediators of this relationship, contrary to expectations.

Common sense says that crimes which are riskier are likely to be the ones with larger rewards, suggesting a positive correlation. This aligns with the notion that risk and reward tend to be positively correlated in the real world (Slovic et al., 2004). Furthermore, in regard to crime, positive correlations between risk and reward should be expected given that qualitative research has demonstrated that the ‘risky’ aspect of crime is in fact part of the thrill (Jacobs, Topalli & Wright, 2003). Research has demonstrated that some offenders enjoy risky crime opportunities (Katz, 1988). Therefore, apprehension risk should be positively correlated with rewards to crime, particularly intrinsic rewards such as the thrill or rush one gets from engaging in crime. However, although direct tests of this relationship do not yet exist in criminology, negative correlations between perceived risks and rewards have been found in several studies and datasets (Pickett & Roche, 2016; McCarthy and Hagan, 2005; Piquero and Tibbetts, 1996; Shulman, Monahan, & Steinberg, 2017). 

\section{CURRENT STUDY}
Given the known generally positive correlation between risk and reward in the world, why then are risk and reward negatively correlated in criminological studies? Do these negative correlations persist when individuals decide to offend? And most importantly, do these perceptions of risk or rewards impact one’s perceptions of the alternative attribute (reward or risk) for criminal decision-making? 

The current study... 


Given the competing hypotheses based on the affect heuristic and more traditional rational choice frameworks, the following hypotheses and alternative hypotheses are tested. 

Hypotheses based on the affect heuristic, suggesting a negative correlation between risk and reward are presented below:  
H1: When risks are greater (smaller), average perceptions of reward will be smaller (greater). 
H2: When social rewards are greater (smaller), average perceptions of risk will be smaller (greater). 
H3: When monetary rewards are greater (smaller), average perceptions of risk will be smaller (greater).  

Alternative hypotheses are presented next which are based on rational decision-making and a positive correlation between risks and reward:
H1A: When risks are greater (smaller), average perceptions of reward will be greater (smaller). 
H2A: When social rewards are greater (smaller), average perceptions of risk will be greater (smaller).
H3A: When monetary rewards are greater (smaller), average perceptions of risk will be greater (smaller). 


\section{DATA}

The data for the current study were collected via the AmeriSpeak survey, a nationally respresentaitve sample of adults in the United States. The sampling frame is a multistage probability sample that represents the U.S. household population. For more information about the sample, see (). The current study was restricted to respondents... 





The study design includes an experimental 4x2 factorial design, resulting in 8 possible experimental units, to assess the effects of risk and reward on perceptions of reward and risk. Of the 4 possible vignettes, respondents were presented with 2 vignettes, each with one experimental manipulation (i.e.., high or low risk or reward). Each respondent received one scenario which manipulated risk (either Vignette A - Drunk Driving or Vignette B - Theft from a Vehicle) and one where social or monetary reward was manipulated (either Vignette C – Fight or Vignette D - Tax Fraud). The scenarios were randomized between respondents and counterbalanced so that half of respondents were presented with the risk scenario prior to the reward scenario and vice versa. The specific wording of the vignettes are presented below. Text in brackets denotes the different manipulations. A summary of the conditions is also provided in the Supplemental Materials.

A pilot study was previously conducted using a sample from Amazon's Mechanical Turk. Following the completion of the pilot, vignettes and questions were reworded to ensure that the experiemental manipulations were successful in manipualting risk and reward. Due to these changes, the pilot data was not pooled with the full sample. However, results of the pilot study are provided in the Supplemental Materials, and results are described in relavent footnotes when results of the main data analysis are described.

\subsection{Experimental Vignettes}

\begin{enumerate}[A]
\item \textbf{Drunk Driving (Social Reward Outcome with Apprehension Risk Manipulated)}
Imagine you drove to meet up with some friends at a bar about 20 minutes from your house. You and your friends have had a great evening singing, dancing, and drinking all night. Its 2am and the bar is shutting down. You are pretty certain you are over the legal limit to drive, but you promised your friends you would drive them home. This is a new group of friends and you really want to be invited out with them again in the future. You know from talking to a colleague who is a retired police officer, that police typically are not out during the week, but instead are usually out on the weekends, though he mentioned that they are almost always out with extra patrols on holiday weekends. It is a [TUESDAY/SATURDAY OVER A HOLIDAY WEEKEND]. 

\item \textbf{Theft from a Vehicle (Monetary Reward Outcome with Apprehension Risk Manipulated)}
Suppose you are walking down the street one evening on your way to hang out with some friends. You pass a nice-looking parked car and slow down to admire it. When you do, you notice that the passenger window is rolled half of the way down and that there appears to be some cash just sitting on the front seat. [YOU HAVEN’T SEEN ANYBODY ELSE AROUND/YOU NOTICED A SECURITY CAMERA A FEW FEET BACK AND YOU CAN SEE SOME POLICE OFFICERS UP AHEAD].

\item \textbf{Fighting (Risk Outcome with Social Reward Manipulated)}
Suppose that you and some friends are at a party at the home of two of your closest friends. The evening is going great and everyone seems to be having a good time. All of a sudden, someone you have never met before gets in your face and starts yelling and shoving you. You have no idea what he is upset about, but he is not letting up and seems to be getting more upset by the minute. You look over at your friends and they are all staring and watching to see what you will do. [YOU KNOW THAT YOUR FRIENDS WOULD BE REALLY IMPRESSED AND THINK IT WAS REALLY COOL IF YOU WERE TO FIGHT THIS GUY/YOU KNOW THAT YOUR FRIENDS, INCLUDING THOSE WHO THREW THE PARTY, WOULD BE EXTREMELY UPSET AND ANGRY AT YOU IF YOU WERE TO FIGHT THIS GUY].


\item \textbf{Tax Fraud (Risk Outcome with Monetary Reward Manipulated)}
It is almost April and it is time to complete your income taxes. You have a close friend who has been a tax accountant for 20 years. They mention that they can help you with your taxes and manipulate some numbers to help save you money. After they go over your taxes, they tell you that they can save you about [\$250/\$1250], but you will have to lie on your tax forms. This is illegal and referred to as tax fraud. 


\subsection{Focal Measures}

After having read each scenario, respondents answered questions regarding their perceived risk of apprehension, overall reward, social costs, monetary reward, intrinsic reward and willingness to offend given the scenario described. 

Perceived risk of apprehension is based on asking respondents, what is the percent chance (or chances out of 100) that they would get caught if you committed the act in the scenario. For the item related to tax fraud, the item asks about being caught by the IRS, while the other items refer to being caught by the police. \footnote{The specific wording of each item is provided in the Supplemental Materials). Respondents were asked to provide a percentage from 0-100. 

Overall reward was measured by asking respondents, how rewarding on a scale of 0-100, with 100 being the most rewarding, would it be if you committed the act in the scenario described. 

Social costs are measured on a five-point Likert scale ranging from 1) very unlikely to 5) extremely likely. Respondents were asked how likely is it that they would you lose respect from their family and friends if they found out that they committed the act in question.

Monetary rewards are measured only for the theft a vehicle scenario (Vignette D) and replace the social costs measure. Monetary reward is defined by a five-item categorical measure. Respondents were asked to select how much money they believe they would get from the car if they took the money. The answer options included \$1-\$25, \$26-\$99, \$100-\$499, \$500-\$999 or \$1000 or more. 

Intrinsic reward is based on the measure employed in the Pathways to Desistance Study (), which asks respondents how much thrill or rush committing the act would be, and relies on a 10-point scale ranging from 0 (No fun or kick at all) to 10 (A great deal of fun or kick). 

Finally, willingness to offen is based on a 5-point Likert scale ranging from 1) very unlikely to 5) extremely likely. Respondents were asked to respond how likely it is they they would commit the act described in each specific scenario. 


\subsection{Predictors}

\subsection{Controls}

Age (years)
Gender
Education
Race/Ethnicity 
Region ((Northeast, Midwest, South, West)
Marital Status
Employment Status
Household Income



\section{ANALYTIC STRATEGY}

\section{RESULTS}


Descriptives on Sample
\insert{T1}

Should I show tables or just figures by manipulation?




\section{DISCUSSION}

The current study contributes to the knowledge base for both criminological theory and policy solutions to crime. Currently, it is accepted that both risks and rewards to crime impact one’s decision to offend, with the more recent research suggesting that rewards play a larger role. The results of the current study suggest...

 If in fact rewards indirectly impact risk perceptions, then this oversight is important for the conceptualization of the role of risk and reward, and also draws into question the ways that these concepts are currently measured.

The current study also adds to our knowledge of how perceptions of arrest risk are formed, an area of study where we know relatively little (Piquero, Paternoster, Pogarsky & Loughran, 2011; Paternoster, 2010). 



\section{CONCLUSION}

\section{REFERENCES}


\section{SUPPLEMENTAL MATERIALS}

\subsection{Question Wording}

\begin{enumerate}
\item What is the percent chance (or chances out of 100) that you would get caught by the [POLICE/ IRS] if you [DROVE HOME/TOOK THIS MONEY/FOUGHT THIS GUY/LIED ON YOUR TAXES]? Please provide a percentage from 0-100. 
\begin{itemize}
\item Open Ended
\end{itemize}
\item How rewarding on a scale of 0-100, with 100 being the most rewarding, would it be if you [DROVE HOME/TOOK THIS MONEY/FOUGHT THIS GUY/LIED ON YOUR TAXES]?
\begin{itemize}
\item Open Ended
\end{itemize}
\item How likely is it that you would you lose respect from your family and friends if they found out that you [DROVE HOME/FOUGHT THIS GUY/LIED ON YOUR TAXES]?
\begin{enumerate}
\item Very Unlikely
\item Unlikely
\item Equally Likely and Unlikely
\item Likely
\item Extremely Likely
\end{enumerate}
\item How much money do you believe you would get from the seat of the car if you TOOK THIS MONEY?   \footnote{This is only asked following Vignette B (Theft from Vehicle) and replaces Question 3 in that condition.}
\begin{enumerate}
\item \$1-\$25
\item \$26-\$99
\item \$100-\$499
\item \$500-\$999
\item \$1000 or more
\end{enumerate}
\item How much thrill or rush would it be if you [DROVE HOME/TOOK THIS MONEY/FOUGHT THIS GUY/LIED ON YOUR TAXES]?
\begin{enumerate}[0]
\item No fun or kick at all  
\item 
\item 
\item
\item 
\item
\item 
\item
\item 
\item A great deal of fun or kick
\end{enumerate}
\item How likely is it that you would [DRIVE HOME/TAKE THIS MONEY/FIGHT THIS GUY/LIE ON YOUR TAXES] in this scenario? 
\begin{enumerate}
\item Very Unlikely
\item Unlikely
\item Equally Likely and Unlikely
\item Likely
\item Extremely Likely
\end{enumerate}
\end{enumerate}


\subsection{Summary of Experimental Conditions}
% Table generated by Excel2LaTeX from sheet 'Sheet1'
\begin{table}[htbp]
  \centering
  \caption{Add caption}
    \begin{tabular}{|c|c|c|c|p{4.975em}|p{8.365em}|}
\cmidrule{2-5}    \rowcolor[rgb]{ .859,  .859,  .859} \multicolumn{1}{|c|}{\multirow{3}[6]{*}{\textbf{Vignette}}} & \multicolumn{1}{c|}{\multirow{3}[6]{*}{\textbf{Focal Dependent Variable}}} & \multicolumn{3}{p{19.95em}|}{\textbf{Manipulation}} & \multirow{3}[6]{*}{\cellcolor[rgb]{ .851,  .851,  .851}\textbf{Condition}} \\
\cmidrule{3-5}    \rowcolor[rgb]{ .859,  .859,  .859}       &       & \multicolumn{1}{p{10em}|}{\textit{Risk}} & \multicolumn{2}{p{9.95em}|}{\textit{Reward}} & \multicolumn{1}{c|}{} \\
\cmidrule{3-5}    \rowcolor[rgb]{ .859,  .859,  .859}       &       &       & \multicolumn{1}{p{4.975em}|}{Social} & Monetary & \multicolumn{1}{c|}{} \\
    \midrule
    \multicolumn{1}{|c|}{\multirow{2}[4]{*}{A (Drunk Driving)}} & \multicolumn{1}{c|}{\multirow{2}[4]{*}{Overall Reward, Social Reward, Intrinsic Reward, Willingness to Offend}} & \multicolumn{1}{p{10em}|}{High} &       & \multicolumn{1}{c|}{} & (1) \\
\cmidrule{3-6}          &       & \multicolumn{1}{p{10em}|}{Low} &       & \multicolumn{1}{c|}{} & (2) \\
    \midrule
    \rowcolor[rgb]{ .929,  .929,  .929} \multicolumn{1}{|c|}{\multirow{2}[4]{*}{B (Theft from Vehicle)}} & \multicolumn{1}{c|}{\multirow{2}[4]{*}{Overall Reward, Monetary Reward, Intrinsic Reward, Willingness to Offend}} & \multicolumn{1}{p{10em}|}{High} &       & \multicolumn{1}{c|}{} & (3) \\
\cmidrule{3-6}    \rowcolor[rgb]{ .929,  .929,  .929}       &       & \multicolumn{1}{p{10em}|}{Low} &       & \multicolumn{1}{c|}{} & (4) \\
    \midrule
    \multicolumn{1}{|c|}{\multirow{2}[4]{*}{C (Fight)}} & \multicolumn{1}{c|}{\multirow{2}[4]{*}{Risk, Willingness to Offend}} &       & \multicolumn{1}{p{4.975em}|}{High} & \multicolumn{1}{c|}{} & (5) \\
\cmidrule{3-6}          &       &       & \multicolumn{1}{p{4.975em}|}{Low} & \multicolumn{1}{c|}{} & (6) \\
    \midrule
    \rowcolor[rgb]{ .929,  .929,  .929} \multicolumn{1}{|c|}{\multirow{2}[4]{*}{D (Tax Fraud)}} & \multicolumn{1}{c|}{\multirow{2}[4]{*}{Risk, Willingness to Offend}} &       &       & High  & (7) \\
\cmidrule{3-6}    \rowcolor[rgb]{ .929,  .929,  .929}       &       &       &       & Low   & (8) \\
    \bottomrule
    \end{tabular}%
  \label{tab:addlabel}%
\end{table}%




\end{document}

